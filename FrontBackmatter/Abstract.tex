%*******************************************************
% Abstract
%*******************************************************
%\renewcommand{\abstractname}{Abstract}
\pdfbookmark[1]{Abstract}{Abstract}
\begingroup
\let\clearpage\relax
\let\cleardoublepage\relax
\let\cleardoublepage\relax

\chapter*{Abstract}

The potential of user-generated sensor data for participatory sensing has motivated the formation of organisations focused on the exploitation of collected information and associated knowledge. 
Given the power and value of both the raw data and the derived knowledge, we advocate an open approach to data and intellectual-property rights. By treating user-generated content as well as derived information and knowledge 
as a common-pool resource, we hypothesise that
all participants can be compensated fairly for their input.

To test this hypothesis, we undertake an extensive review of experimental, commercial and social participatory-sensing applications, from which we identify that a decentralised, community-oriented governance model is required
to support this open approach. We show that the Institutional Analysis and Design framework as introduced by Elinor Ostrom, in conjunction
with a framework for self-organising electronic institutions, can be used to give both an architectural
and algorithmic base for the necessary governance model, in terms of operational and collective choice rules
specified in computational logic.

As a basis for understanding the effect of governance on these applications, we develop a testbed which joins our logical formulation of the knowledge commons with a generic model of the participatory-sensing problem.
This requires a multi-agent platform for the simulation of autonomous and dynamic agents, and a method of executing the logical calculus in which our electronic institution is specified. 
To this end, firstly, we develop a general purpose, high performance platform for multi-agent based simulation, Presage2. 
Secondly, we propose a method for translating event-calculus axioms into rules compatible with business rule engines, and provide an implementation for JBoss Drools along with a suite of modules for electronic institutions.

Through our simulations we show that, when building electronic institutions for managing participatory sensing as a knowledge commons, proper enfranchisement of agents (as outlined in Ostrom's work) is key to striking a balance between endurance, fairness and reduction of greedy behaviour. We conclude with a set of guidelines for engineering knowledge commons for the next generation of participatory-sensing applications.

%For this testbed, a general-purpose multi-agent simulation platform, Presage2, is developed, and a rule-engine for electronic institutions which can execute our logic specification. 
%Problem Statement
%These organisations often have centralised governance and/or a commercial imperative which may create an inequitable exchange (of data for services) for the data providers (i.e. the users).
%Approach
%Considering data and knowledge as a common-pool resource, or knowledge commons, we review the management of knowledge in social and socio-technical systems in order to determine how organisational structure and rules affect individuals and their use of shared resources. We formulate a generalisation of the participatory sensing problem using our findings to present a framework for the management of data and knowledge in open systems.
%In this work we study where collective, self-organising approaches to data and knowledge management has been successful and assess whether this organisational structure could be created and maintained by agents in a technical system.
%In this work we study the type of organisations that use a collective, self-organising approach to data and knowledge management and assess their success. We study whether this organisational structure can be translated for technical systems created ...
%In this work we study organisations, such as Wikipedia, that have a collective, self-organising approach to knowledge management and analyse their structure and rules. We then use these findings to assess whether this organisational structure could be created for and maintained by technical systems. 
%Results
%We present an instantiation of the Institutional Analysis and Development Framework with respect to participatory sensing problems and a formalisation of rules we see in the studied organisations.
%Conclusions
%This provides a basis for understanding firstly, how a self-organising institution can be created and maintained by autonomous agents in a technical system, and secondly, what effect the organisational structure and rules will have on both the organisation and individuals. 
% Jeremy one-liner
%The result is a user-centred design of self-governing institutions for participatory sensing based applications, which deliver a more equitable distribution and exchange for the `grassroots' data providers on the edge of the network.

%Contributions:
%\begin{itemize}
%\item Review of Knowledge Commons for Participatory sensing
%\item Simulation platform for multi-agent systems which can be used for principled operationalisation.
%\item A method of transforming a logical calculus describing institutional rules into one which can be executed using a high performance business rule engine and an implementation in Drools.
%\item Experimental demonstration of simulation platform and rule engine to show the value of Ostrom's principle 3 when applied to participatory sensing.
%\end{itemize}
%
%So what?:
%\begin{itemize}
%\item Re-usable, general purpose, high performance simulation platform for multi-agent systems.
%\item Re-usable, modular, customisable, high performance rule engine for electronic institutions.
%\item Guidelines for supply of institutions for participatory sensing.
%\end{itemize}

\endgroup			

\vfill