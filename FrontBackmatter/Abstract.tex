%*******************************************************
% Abstract
%*******************************************************
%\renewcommand{\abstractname}{Abstract}
\pdfbookmark[1]{Abstract}{Abstract}
\begingroup
\let\clearpage\relax
\let\cleardoublepage\relax
\let\cleardoublepage\relax

\chapter*{Abstract}
The proliferation of sensor networks, mobile and pervasive computing has provided the technological push
for a new class of participatory sensing applications, based on sensing and aggregating user-generated
content, and transforming it into knowledge. However, given the power and value of both the raw data
and the derived knowledge, to ensure that the generators are commensurate beneficiaries, we advocate an
open approach to the data and intellectual property rights by treating user-generated content, and derived
information and knowledge as a common-pool resource. 

In this paper, we undertake an extensive review
of experimental, commercial and social participatory sensory applications, from which we identify that a
decentralised, community-oriented governance model is required to support this approach. Furthermore,
we show that Ostrom's Institutional Analysis and Design framework, in conjunction with a framework
for self-organising electronic institutions, can be used to give both an architecture and algorithmic base
for the requisite governance model, in terms of operational and collective choice rules specified in
computational logic. This provides, we believe, the foundations for engineering knowledge commons
for the next generation of participatory sensing applications, in which the data generators are also the
primary beneficiaries.


\endgroup			

\vfill