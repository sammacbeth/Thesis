%!TEX root =../MacbethThesis.tex

\chapter{Event Calculus Benchmarking}

\section{RONR Implementation in Prolog}\label{sec:ronrcode}

\begin{prologinline}[caption=Implementation of RONR in Prolog]
:- dynamic
    happens/2.

/****************************************
 * SYNTAX OF ACTIONS                    *
 *                                      *
 * open_session(Agent, Session)         *
 * close_session(Agent, Session)        *
 * propose(Agent, Motion)               *
 * second(Agent, Motion)                *
 * open_ballot(Agent, Motion)           *
 * close_ballot(Agent, Motion)          *
 * vote(Agent, Motion, aye)             *
 * vote(Agent, Motion, nay)             *
 * declare(Agent, Motion, carried)      *
 * declare(Agent, Motion, not_carried)  *
 ***************************************/                          

/************
 * SESSIONS *
 ***********/
initiates( open_session(C,S), sitting(S)=true, T ) :-
	holdsAt( pow(C, open_session(C,S))=true, T ).

initiates( open_session(C,S), resolutions=([],[]), T ) :-
	holdsAt( pow(C, open_session(C,S))=true, T ).

initiates( close_session(C,S), sitting(_)=false, T ) :-
	holdsAt( pow(C, close_session(C,S))=true, T ).

/********************
 * STATE of MOTIONS *
 *******************/

% -----      pAgent       sAgent       chair      chair      chair
% -----     propose       second   open_ballot close_ballot declare
% ----- (null) --> proposed --> seconded --> voting --> voted --> {carried | not_carried}

initiates( propose(A,M), status(M)=proposed, T ) :-
	holdsAt( pow(A, propose(A,M))=true, T ).

initiates( second(B,M), status(M)=seconded, T ) :-
	holdsAt( pow(B, second(B,M))=true, T ).

initiates( open_ballot(C,M), status(M)=voting(T), T ) :-
	holdsAt( pow(C, open_ballot(C,M))=true, T ).

initiates( close_ballot(C,M), status(M)=voted, T ) :-
	holdsAt( pow(C, close_ballot(C,M))=true, T ).

initiates( declare(C,M,carried), status(_)=null, T ) :-
	holdsAt( pow(C, declare(C,M,_))=true, T ).

initiates( declare(C,M,not_carried), status(_)=null, T ) :-
	holdsAt( pow(C, declare(C,M,_))=true, T ).

initiates( declare(C,M,carried), resolutions=([M|Ms],Ns), T ) :-
	holdsAt( pow(C, declare(C,M,_))=true, T ),
	holdsAt( resolutions=(Ms,Ns), T ).

initiates( declare(C,M,not_carried), resolutions=(Ms,[M|Ns]), T ) :-
	holdsAt( pow(C, declare(C,M,_))=true, T ),
	holdsAt( resolutions=(Ms,Ns), T ).

/***********************
 * INSTITUTIONAL POWER *
 ***********************/
holdsAt( pow(C, open_session(C,S))=true, T ) :-
	holdsAt( sitting(S)=false, T ),
	holdsAt( role_of(C,chair)=true, T ).

holdsAt( pow(A, propose(A,M))=true, T ) :-
	holdsAt( status(M)=null, T ),
	holdsAt( role_of(A,proposer)=true, T ).

holdsAt( pow(B, second(B,M))=true, T ) :-
	holdsAt( status(M)=proposed, T ),
	holdsAt( role_of(B,seconder)=true, T ).

holdsAt( pow(C, open_ballot(C,M))=true, T ) :-
	holdsAt( status(M)=seconded, T ),
	holdsAt( role_of(C,chair)=true, T ).

holdsAt( pow(V, vote(V,M,_))=true, T ) :-
	holdsAt( status(M)=voting(_), T ),
	holdsAt( role_of(V,voter)=true, T ),
	\+ holdsAt( role_of(V,chair)=true, T ),
	holdsAt( voted(V,M)=nil, T ).

holdsAt( pow(C, close_ballot(C,M))=true, T ) :-
	holdsAt( status(M)=voting(Te), T ), Te < T,
	holdsAt( role_of(C,chair)=true, T ).

holdsAt( pow(C, declare(C,M,_))=true, T ) :-
	holdsAt( status(M)=voted, T ),
	holdsAt( role_of(C,chair)=true, T ).

holdsAt( pow(C, close_session(C,S))=true, T ) :-
	holdsAt( sitting(S)=true, T ),
	holdsAt( role_of(C,chair)=true, T ).

holdsAt( pow(Agent, Action) = false, T ) :-
	\+ holdsAt( pow(Agent, Action) = true, T ).

/*****************************
 * ROLE ASSIGNMENT (SORT OF) *
 ****************************/
initiates( propose(A,M), role_of(B,seconder)=true, T ) :-
	holdsAt( pow(A, propose(A,M))=true, T ),
	holdsAt( qualifies(B,seconder)=true, T ),
	A \= B.

initiates( second(B1,M), role_of(B2,seconder)=false, T ) :-
	holdsAt( pow(B1, second(B1,M))=true, T ),
    holdsAt( qualifies(B2,seconder)=true, T ).

initiates( open_session(C,M), role_of(A,proposer)=true, T ) :-
	holdsAt( pow(C, open_session(C,M))=true, T ),
	holdsAt( qualifies(A,proposer)=true, T ).

initiates( close_session(C,M), role_of(A,proposer)=false, T ) :-
	holdsAt( pow(C, close_session(C,M))=true, T ),
	holdsAt( qualifies(A,proposer)=true, T ).

/*****************************
 * VOTING and COUNTING VOTES *
 ****************************/

% ----- open ballot and initiate votes to (0,0)
% ----- vote for (F,A) --> (F1,A)
% ----- vote against (F,A) --> (F,A1)
% ----- power to vote removed by either act of voting or chair closing the ballot

initiates( open_ballot(C,M), votes(M)=(0,0), T ) :-
	holdsAt( pow(C, open_ballot(C,M))=true, T ).

initiates( open_ballot(C,M), voted(V,M)=nil, T ) :-
    holdsAt( pow(C, open_ballot(C,M))=true, T ),
	holdsAt( role_of(V,voter)=true, T ).

initiates( vote(V,M,aye), votes(M)=(F1,A), T ) :-
	holdsAt( pow(V, vote(V,M,_))=true, T ),
	holdsAt( votes(M)=(F,A), T ),
	F1 is F + 1.

initiates( vote(V,M,aye), voted(V,M)=aye, T ) :-
	holdsAt( pow(V, vote(V,M,_))=true, T ).

initiates( vote(V,M,nay), votes(M)=(F,A1), T ) :-
	holdsAt( pow(V, vote(V,M,_))=true, T ),
	holdsAt( votes(M)=(F,A), T ),
	A1 is A + 1.

initiates( vote(V,M,nay), voted(V,M)=nay, T ) :-
	holdsAt( pow(V, vote(V,M,_))=true, T ).

initially( status(_) = null ).
initially( sitting(_) = false ).
\end{prologinline}