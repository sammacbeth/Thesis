%!TEX root =../MacbethThesis.tex

\chapter{Voting Protocol Specifications}

Here we list the specifications of the voting protocol from \ac{RONR} used in
the benchmarks in \autoref{ch:droolseinst}, as well as the voting protocol used 
in the experiments presented in \autoref{ch:results}. We have firstly a Prolog
implementation from the original paper of \citet{Pitt2005a}. Secondly, we have
the translation of this specification into Drools. This specification only
contains the Drools code, and is missing the associated Java classes required
to execute it. These classes, along with associated code for testing the
specification can be found in the Drools-EInst source
repository\footnote{https://github.com/sammacbeth/electronic-institutions}. Thirdly,
the voting protocol for our experiments is described in terms of the actions and fluents used. 
The full Drools specification is also available in the Drools-EInst repository.

\section{RONR Voting Protocol Implementation in Prolog}\label{sec:ronrcode}

\begin{prologinline}
:- dynamic
    happens/2.

/****************************************
 * SYNTAX OF ACTIONS                    *
 *                                      *
 * open_session(Agent, Session)         *
 * close_session(Agent, Session)        *
 * propose(Agent, Motion)               *
 * second(Agent, Motion)                *
 * open_ballot(Agent, Motion)           *
 * close_ballot(Agent, Motion)          *
 * vote(Agent, Motion, aye)             *
 * vote(Agent, Motion, nay)             *
 * declare(Agent, Motion, carried)      *
 * declare(Agent, Motion, not_carried)  *
 ***************************************/                          

/************
 * SESSIONS *
 ***********/
initiates( open_session(C,S), sitting(S)=true, T ) :-
	holdsAt( pow(C, open_session(C,S))=true, T ).

initiates( open_session(C,S), resolutions=([],[]), T ) :-
	holdsAt( pow(C, open_session(C,S))=true, T ).

initiates( close_session(C,S), sitting(_)=false, T ) :-
	holdsAt( pow(C, close_session(C,S))=true, T ).

/********************
 * STATE of MOTIONS *
 *******************/

% -----      pAgent       sAgent       chair      chair      chair
% -----     propose       second   open_ballot close_ballot declare
% ----- (null) --> proposed --> seconded --> voting --> voted --> {carried | not_carried}

initiates( propose(A,M), status(M)=proposed, T ) :-
	holdsAt( pow(A, propose(A,M))=true, T ).

initiates( second(B,M), status(M)=seconded, T ) :-
	holdsAt( pow(B, second(B,M))=true, T ).

initiates( open_ballot(C,M), status(M)=voting(T), T ) :-
	holdsAt( pow(C, open_ballot(C,M))=true, T ).

initiates( close_ballot(C,M), status(M)=voted, T ) :-
	holdsAt( pow(C, close_ballot(C,M))=true, T ).

initiates( declare(C,M,carried), status(_)=null, T ) :-
	holdsAt( pow(C, declare(C,M,_))=true, T ).

initiates( declare(C,M,not_carried), status(_)=null, T ) :-
	holdsAt( pow(C, declare(C,M,_))=true, T ).

initiates( declare(C,M,carried), resolutions=([M|Ms],Ns), T ) :-
	holdsAt( pow(C, declare(C,M,_))=true, T ),
	holdsAt( resolutions=(Ms,Ns), T ).

initiates( declare(C,M,not_carried), resolutions=(Ms,[M|Ns]), T ) :-
	holdsAt( pow(C, declare(C,M,_))=true, T ),
	holdsAt( resolutions=(Ms,Ns), T ).

/***********************
 * INSTITUTIONAL POWER *
 ***********************/
holdsAt( pow(C, open_session(C,S))=true, T ) :-
	holdsAt( sitting(S)=false, T ),
	holdsAt( role_of(C,chair)=true, T ).

holdsAt( pow(A, propose(A,M))=true, T ) :-
	holdsAt( status(M)=null, T ),
	holdsAt( role_of(A,proposer)=true, T ).

holdsAt( pow(B, second(B,M))=true, T ) :-
	holdsAt( status(M)=proposed, T ),
	holdsAt( role_of(B,seconder)=true, T ).

holdsAt( pow(C, open_ballot(C,M))=true, T ) :-
	holdsAt( status(M)=seconded, T ),
	holdsAt( role_of(C,chair)=true, T ).

holdsAt( pow(V, vote(V,M,_))=true, T ) :-
	holdsAt( status(M)=voting(_), T ),
	holdsAt( role_of(V,voter)=true, T ),
	\+ holdsAt( role_of(V,chair)=true, T ),
	holdsAt( voted(V,M)=nil, T ).

holdsAt( pow(C, close_ballot(C,M))=true, T ) :-
	holdsAt( status(M)=voting(Te), T ), Te < T,
	holdsAt( role_of(C,chair)=true, T ).

holdsAt( pow(C, declare(C,M,_))=true, T ) :-
	holdsAt( status(M)=voted, T ),
	holdsAt( role_of(C,chair)=true, T ).

holdsAt( pow(C, close_session(C,S))=true, T ) :-
	holdsAt( sitting(S)=true, T ),
	holdsAt( role_of(C,chair)=true, T ).

holdsAt( pow(Agent, Action) = false, T ) :-
	\+ holdsAt( pow(Agent, Action) = true, T ).

/*****************************
 * ROLE ASSIGNMENT (SORT OF) *
 ****************************/
initiates( propose(A,M), role_of(B,seconder)=true, T ) :-
	holdsAt( pow(A, propose(A,M))=true, T ),
	holdsAt( qualifies(B,seconder)=true, T ),
	A \= B.

initiates( second(B1,M), role_of(B2,seconder)=false, T ) :-
	holdsAt( pow(B1, second(B1,M))=true, T ),
    holdsAt( qualifies(B2,seconder)=true, T ).

initiates( open_session(C,M), role_of(A,proposer)=true, T ) :-
	holdsAt( pow(C, open_session(C,M))=true, T ),
	holdsAt( qualifies(A,proposer)=true, T ).

initiates( close_session(C,M), role_of(A,proposer)=false, T ) :-
	holdsAt( pow(C, close_session(C,M))=true, T ),
	holdsAt( qualifies(A,proposer)=true, T ).

/*****************************
 * VOTING and COUNTING VOTES *
 ****************************/

% ----- open ballot and initiate votes to (0,0)
% ----- vote for (F,A) --> (F1,A)
% ----- vote against (F,A) --> (F,A1)
% ----- power to vote removed by either act of voting or chair closing the ballot

initiates( open_ballot(C,M), votes(M)=(0,0), T ) :-
	holdsAt( pow(C, open_ballot(C,M))=true, T ).

initiates( open_ballot(C,M), voted(V,M)=nil, T ) :-
    holdsAt( pow(C, open_ballot(C,M))=true, T ),
	holdsAt( role_of(V,voter)=true, T ).

initiates( vote(V,M,aye), votes(M)=(F1,A), T ) :-
	holdsAt( pow(V, vote(V,M,_))=true, T ),
	holdsAt( votes(M)=(F,A), T ),
	F1 is F + 1.

initiates( vote(V,M,aye), voted(V,M)=aye, T ) :-
	holdsAt( pow(V, vote(V,M,_))=true, T ).

initiates( vote(V,M,nay), votes(M)=(F,A1), T ) :-
	holdsAt( pow(V, vote(V,M,_))=true, T ),
	holdsAt( votes(M)=(F,A), T ),
	A1 is A + 1.

initiates( vote(V,M,nay), voted(V,M)=nay, T ) :-
	holdsAt( pow(V, vote(V,M,_))=true, T ).

initially( status(_) = null ).
initially( sitting(_) = false ).
\end{prologinline}

\section{RONR Voting Protocol Implementation in Drools}\label{sec:ronrdrools}

\begin{droolsinline}
/************
 * SESSIONS *
 ***********/
rule "Open Session"
	when
		$open : OpenSession($a : actor, $i : inst, $name : name, $t : t, valid == false)
		Pow(actor == $a, this.matches($open))
		not( exists( Session(inst == $i, name == $name, sitting == true) ) )
	then
		insert(new Session($i,  $name, true));
		modify($open) {
			setValid(true);
		}
end

rule "Close session"
	when
		$close : CloseSession($a : actor, $i : inst, $name : name, $t : t, valid == false)
		$sesh : Session(inst == $i, name == $name)
		Pow(actor == $a, this.matches($close))
	then
		modify($sesh) {
			setSitting(false);
		}
		modify($close) {
			setValid(true);
		}
end

query getSession(Institution i, String name)
	Session(inst == i, name == name, session : this)
end

/********************
 * STATE of MOTIONS *
 *******************/
rule "Propose"
 	when
 		$prop : Propose($a : actor, $m : motion, valid == false)
 		Pow(actor == $a, this.matches($prop))
 		not( exists( Motion(session == $m.session, name == $m.name ) ) )
 	then
 		$m.setStatus(Motion.Status.Proposed);
 		insert($m);
 		modify($prop) {
 			setValid(true);
 		}
end
 
rule "Second"
	when
		$sec : Second($a : actor, $m : motion, valid == false)
		Pow(actor == $a, this.matches($sec))
	then
		modify($m) {
			setStatus(Motion.Status.Seconded);
		}
		modify($sec) {
			setValid(true);
		}
end

rule "Open ballot"
	when
		$open : OpenBallot($a : actor, $m : motion, valid == false)
		Pow(actor == $a, this.matches($open))
	then
		modify($m) {
			setStatus(Motion.Status.Voting),
			setVoting($open.getT());
		}
		modify($open) {
			setValid(true);
		}
end

rule "Close ballot"
	when
		$close : CloseBallot($a : actor, $m : motion, valid == false)
		Pow(actor == $a, this.matches($close))
	then
		modify($m) {
			setStatus(Motion.Status.Voted);
		}
		modify($close) {
			setValid(true);
		}
end

rule "Declare carried"
	when
		$decl : Declare($a : actor, $m : motion, status == Motion.Status.Carried)
		Pow(actor == $a, this.matches($decl))
	then
		modify($m) {
			setStatus(Motion.Status.Carried);
		}
		modify($m.getSession()) {
			addCarried($m);
		}
end

rule "Declare not carried"
	when
		$decl : Declare($a : actor, $m : motion, status == Motion.Status.NotCarried)
		Pow(actor == $a, this.matches($decl))
	then
		modify($m) {
			setStatus(Motion.Status.NotCarried);
		}
		modify($m.getSession()) {
			addNotCarried($m);
		}
end
 
query getMotion(Institution i, String sesh, String name)
 	Session(inst == i, name == sesh, session : this)
	Motion(name == name, session == session, motion : this)
end
 
/***********************
 * INSTITUTIONAL POWER *
 ***********************/
rule "Pow open session"
	when
		RoleOf($c : actor, $i : inst, role == "chair")
	then
		insertLogical(new Pow($c, new OpenSession($c, $i, null)));
end

rule "Pow propose motion"
	when
		RoleOf($p : actor, $i : inst, role == "proposer")
		$sesh : Session(inst == $i, sitting == true)
	then
		insertLogical(new Pow($p, new Propose($p, new Motion($sesh, null))));
end

rule "Pow second motion"
	when
		RoleOf($p : actor, $i : inst, role == "seconder")
		$sesh : Session(inst == $i, sitting == true)
		$m : Motion(session == $sesh, status == Motion.Status.Proposed)
	then
		insertLogical(new Pow($p, new Second($p, $m)));
end

rule "Pow open ballot"
	when
		RoleOf($c : actor, $i : inst, role == "chair")
		$sesh : Session(inst == $i, sitting == true)
		$m : Motion(session == $sesh, status == Motion.Status.Seconded)
	then
		insertLogical(new Pow($c, new OpenBallot($c, $m)));
end

rule "Pow vote"
	when
		RoleOf($a : actor, $i : inst, role == "voter")
		$sesh : Session(inst == $i, sitting == true)
		$m : Motion(session == $sesh, status == Motion.Status.Voting)
		not( RoleOf(actor == $a, inst == $i, role == "chair") )
		not( Vote(actor == $a, motion == $m, valid == true) )
	then
		insertLogical(new Pow($a, new Vote($a, $m, null)));
end

rule "Pow close ballot"
	when
		RoleOf($c : actor, $i : inst, role == "chair")
		$sesh : Session(inst == $i, sitting == true)
		T($t : t)
		$m : Motion(session == $sesh, status == Motion.Status.Voting, voting < $t)
	then
		insertLogical(new Pow($c, new CloseBallot($c, $m)));
end

rule "Pow declare"
	when
		RoleOf($c : actor, $i : inst, role == "chair")
		$sesh : Session(inst == $i, sitting == true)
		$m : Motion(session == $sesh, status == Motion.Status.Voted)
	then
		insertLogical(new Pow($c, new Declare($c, $m, null)));
end

rule "Pow close session"
	when
		RoleOf($c : actor, $i : inst, role == "chair")
		Session(inst == $i, sitting == true, $name : name)
	then
		insertLogical(new Pow($c, new CloseSession($c, $i, $name)));
end

/*****************************
 * ROLE ASSIGNMENT (SORT OF) *
 ****************************/

rule "Assign seconders"
	when
		T($t : t)
		Propose($a : actor, $i : inst, valid == true, t == $t)
		Qualifies($b : actor, role == "seconder", actor != $a)
	then
		insert( new RoleOf($b, $i, "seconder") );
end

rule "Unassign seconders"
	when
		T($t : t)
		Second($i : inst, valid == true, t == $t)
		$r : RoleOf(role == "seconder", inst == $i)
	then
		retract($r);
end

rule "Assign proposers"
	when
		T($t : t)
		OpenSession($i : inst, valid == true, t == $t)
		Qualifies($a : actor, role == "proposer")
	then
		insert( new RoleOf($a, $i, "proposer") );
end

rule "Unassign proposers"
	when
		T($t : t)
		CloseSession($i : inst, valid == true, t == $t)
		$r : RoleOf(role == "proposer", inst == $i)
	then
		retract($r);
end

/*****************************
 * VOTING and COUNTING VOTES *
 ****************************/

rule "Aye Vote"
	when
		$v : Vote($a : actor, $i : inst, $m : motion, vote == Vote.Choice.AYE, valid == false)
		Pow(actor == $a, this.matches($v))
	then
		modify($m) {
			addAye();	
		}
		modify($v) {
			setValid(true);
		}
end

rule "Nay Vote"
	when
		$v : Vote($a : actor, $i : inst, $m : motion, vote == Vote.Choice.NAY, valid == false)
		Pow(actor == $a, this.matches($v))
	then
		modify($m) {
			addNay();
		}
		modify($v) {
			setValid(true);
		}
end
\end{droolsinline}

\section{Alternative Voting Protocol Specification}\label{sec:votingspec}

Here we state the actions and fluents of the voting specification used in \autoref{ch:results}.

There are four actions in the protocol:
\begin{itemize}
\item \texttt{OpenBallot(actor, inst, issue)}: This action opens a ballot on an issue \texttt{issue} in the institution \texttt{inst}, provided the agent \texttt{actor} is empowered to do so.
\item \texttt{CloseBallot(actor, inst, ballot)}: This action closes the open ballot \texttt{ballot} in the institution \texttt{inst}, provided the agent \texttt{actor} is empowered to do so.
\item \texttt{Vote(actor, inst, ballot, vote)}: This action specifies a vote of \texttt{vote} for the agent \texttt{actor} on ballot \texttt{ballot} in the institution \texttt{inst}, provided the agent \texttt{actor} is empowered to vote on this ballot.
\item \texttt{Declare(actor, inst, ballot, winner)}: This action declares the winning candidate of the ballot \texttt{ballot} in institution \texttt{inst}. The ballot must have been closed, and the agent \texttt{actor} must be empowered to declare the result of the ballot.
\end{itemize}

The state of votes and issues are represented by \texttt{Issue}s and \texttt{Ballot}s.

An \texttt{Issue} represents a topic about which a ballot may be called. It contains the following attributes:
\begin{itemize}
\item The institution to which this issue belongs, \texttt{institution};
\item the name of the issue, \texttt{name};
\item a set of agent roles who are empowered to open, close and declare results of ballots on this issue, \texttt{cfvroles};
\item a set of agent rules who are empowered to vote on this issue, \texttt{voteroles};
\item the method with which voters should submit their votes (\eg\ single vote, preference list), \texttt{method};
\item a set of allowable values or candidates which can be submitted in a vote, \texttt{votevalues};
\item the winner determination method for votes on this issue (\eg\ plurality, borda count etc.), \texttt{wdm}.
\end{itemize}

A \texttt{Ballot} represents a collection of votes on an issue. It contrains the following attributes:
\begin{itemize}
\item The status of the ballot: open or closed, \texttt{status};
\item the issue to which this ballot belongs, \texttt{issue};
\item the timestep in which this ballot was opened, \texttt{started};
\item the set of agent roles who are empowered to vote in this ballot, \texttt{voteRoles};
\item the set of allowable values for agents to vote on, \texttt{options};
\item the winner determination method for this vote, \texttt{wdm};
\item the agent who closed the ballot (used to specify who is obliged to declare the result), \texttt{closedBy}.
\end{itemize}

