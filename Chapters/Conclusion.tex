%!TEX root =../MacbethThesis.tex
\acresetall
\chapter{Summary, Conclusion and Further Work}\label{ch:conc}

\lettrine[lines=3]{I}{n} this thesis, we have described the importance of
governance to participatory-sensing applications, and how algorithmic
governance can be deployed, in line with the theory of the knowledge commons,
to lead to a more equitable outcome for the users of these applications.

In \autoref{ch:kc} we reviewed governance in participatory-sensing
applications, finding predominantly centralisation and lack of 
user-enfranchisement. Having looked at the theory of the knowledge commons, from
the social sciences, we saw that similar systems based on the exchange of
information and knowledge could be managed as a shared resource, and the
principles defined by \citet{Ostrom1990} could be applied to increase the
likelihood of success of this endeavour. Thus we argued that the
characteristics of participatory sensing allow it to be seen as a knowledge
commons, and that managing as such could address the problems we identified in
current applications.

We presented a formalisation of participatory sensing as a provision and
appropriation system in \autoref{sec:iad}. This allowed us to do a thorough
analysis of the problem using Ostrom's \ac{IAD} framework, and to link this
system to an axiomatisation of Ostrom's principles, by \citet{Pitt2012b}. This
led to a derivation of the latter's \ac{EC} axioms for the purpose of the
knowledge commons, and participatory sensing in particular.

The problem of implementing the system and axioms described in
\autoref{sec:iad} posed a significant technical challenge. In
\autoref{ch:presage} a simulation platform for this purpose, Presage2, was
described. We followed with a discussion of the specification and execution of
electronic institutions in \autoref{ch:droolseinst}. We presented a method for
translation of \ac{EC} axioms for use with the JBoss Drools rule engine to
enable fast runtime execution of \ac{EC} specifications, and a reference
implementation with accompanying libraries, Drools-EInst.

In \autoref{ch:results} we proposed a model of participatory sensing as a
reinforcement learning problem. We implemented this model using Presage2, then
integrated this model with an institution specification, based on the axioms
given in \autoref{sec:iad}, and implemented using Drools-EInst. Using this
simulation model we ran a series of experiments which give us empirical
evidence of the benefits of Ostrom's principles for management of a knowledge
commons in participatory sensing.

This thesis therefore contributes the following:

\begin{itemize}
\item A review of governance in participatory-sensing applications, identifying a lack of governance consideration and user enfranchishment.
\item An analysis of participatory sensing as a knowledge commons, using the \ac{IAD} framework, in conjunction with a framework for self-organising electronic institutions, which provides an architectural and algorithmic basis for governance of a knowledge commons.
\item A general purpose simulation platform for agent-based simulation and modelling, Presage2, suitable of the principled operationalisation of a model of the participatory-sensing knowledge commons.
\item A method of translating \acl{EC} into business rules, and an implementation for specification of electronic institutions, Drools-EInst, along with a suite of modules, with which we can implement a specification for a self-organising knowledge commons.
\item An experimental model of the management of participatory sensing as a knowledge commons, through which we generate an experimental validation of the problem of supply of institutions, and the importance of proper enfranchisement of the data providers for equity in participatory sensing.
\end{itemize}

\section{Conclusions}

This work represents the first steps to a design toolkit for the digital
knowledge commons. Much like the \ac{FOSS} model provides a framework for
starting an open-source software project, we are providing a framework for
providing provision and appropriation systems for participatory sensing.
\citet{Ostrom1990} described the problem of providing suitable institutions,
which tends to lead to simpler to manage centralised solutions. Our analysis
demonstrates why this is the case: there are many factors to consider which,
if not handled correctly can lead to a failure of the institution.

Our guidelines for creating an enduring and equitable knowledge commons are
based on Ostrom's principles, and their axiomatisation by \citet{Pitt2012b}.
The latter's formalisation deals with an exogenous resource-allocation system
with highly excludable resources, while the knowledge commons is an
endogenous provision and appropriation system with no excludability. This
means that the specifications differ significantly, and we can no longer
assume some of the previous conclusions with respect to Ostrom's principles.
\citet{Pitt2012b} described the effect of six of of Ostrom's principles, and this was extended at all 8 by \citet{Schaumeier2013}. These benefits are shown in \autoref{tab:principlebenefit}.

\begin{table}[ht]
\centering
\caption[Benefits of Ostrom's principles for resource allocation]{Benefits of Ostrom's principles for resource allocation~\citep{Schaumeier2013}}\label{tab:principlebenefit}
\begin{tabularx}{\textwidth}{lp{4cm}X}
\multicolumn{2}{c}{Principle} & Benefit \\
\hline
1 & Clearly defined boundaries & robustness to intentional violation by outsiders/overpopulation \\
2 & Provision and appropriation rules & robustness to environmental variation \\
\multirow{2}{*}{3} & \multirow{2}{*}{Collective-choice} & robustness to environmental variation \\
 & & robustness to `unfair' behaviour \\
4 & Monitoring & robustness to non-compliant behaviour \\
\multirow{2}{*}{5} & \multirow{2}{*}{Graduated sanctions} & mitigation of intentional violation \\
 & & tolerance of unintentional violation \\
6 & Conflict-resolution mechanisms & repair of unintentional violation \\
7 & Minimal recognition of rights to organise & robustness to arbitrariness/despotism \\
8 & Nested enterprises & robustness to underpopulation \\
\end{tabularx}
\end{table}

We showed how Principle 1 can be addressed for a purely digital resource via
role-based access control. This establishes institutionalised power to
contribute and extract from the resource, effectively prevent outside access.
This does, however, rely on the security of the authentication and
protection mechanisms used to provide this access control. 
% The recent trend of
% high profile security leaks in software demonstrate that this is not an easy
% task.

Our experiments in supplying institutions showed that provision and
appropriation rules play an important role in providing robustness to
environmental conditions. Therefore, with significantly different rules, and
different resource characteristics we observed the same effect as in
\citet{Pitt2012b} for Principle 2. Furthermore, we also showed that Principle 2
can mitigate greed by making the dominant strategy a sustainable one.

In participatory sensing we saw that covering the costs of measuring is an
important aspect in ensuring all agents contribute, and providing a balance
between the cost of raw and derived information was key to sustainability and
equity.

Principle 3 provides further robustness to environmental variation by allowing
the aforementioned rules to be modified. \citet{Schaumeier2013} also shows
that this principle reduces `unfair' behaviour, according to various fairness
metrics. We show, similarly, that we can reduce inequitable outcomes through
this principle, which, for some definitions of fairness, is the same
result\footnote{\citet{Schaumeier2013} initially measures fairness using the
concept of utilitarian social welfare, which is equivalent to our `equity'
metric. They later explore alternative, individual and self-aware fairness
measures.}. Furthermore, we showed that this principle is able to mitigate the
effect of agents with greedy strategies gaining self-benefit. This is an
important property, as over time it will lead to greater levels of compliance
and cooperation in a system as agents' strategies evolve. This property was
demonstrated by \citet{Axelrod1984}---that agents will evolve a socially
optimal strategy if they cannot extract individual benefit in the long run
from greedy decisions.

We discussed Principle 4 in \autoref{sec:iad}, noting that if institutional
actions are logged, analysis of this log can detect institutional violations.
Again, having a digital resource with access control largely prevents
malicious extraction of it, which was the primary target of this principle in
\possessivecite{Ostrom1990} work. However, other violations such as
provisioning false information have yet to be explored here. We expect the
approaches to monitoring given are likely to be appropriate. % TODO monitoring techniques.

Principles 5 and 6 are general purpose and do not depend on the specifics of
the system characteristics, beyond the specific sanctions given. Therefore we
expect the conclusions of \citet{Pitt2012b,Schaumeier2013} to hold: that
together they provide mitigation of non-compliant behaviour, while being able
to self-correct for when punishments are incorrectly given.

We discussed how principle 7 could be applied in the context of a knowledge
commons. In the knowledge commons resources are heterogeneous, and attributed
to an individual or group of individuals, therefore the rights given through
copyright law may contradict the institutional rules. The requirement of
permissive licensing of provisioned content can be used to prevent such
conflicts without taking away other important rights of the individuals.
However, this hypothesis remains to be tested.

Thus, this work has provided a study of the similarities and differences in
application of Ostrom's principles between a resource allocation system, where
resource provision is exogenous and appropriation is subtractive, and a
provision and appropriation system, in which provision is endogenous and
appropriation non-subtractive. The common-pool resource abstraction is
applicable to both, but specifics of suitable rules vary with the types of
actions performed, which patterns of interaction are to be encouraged, and
which are to be discouraged. We find that the \ac{IAD} framework is a
particularly useful tool for determining these specifics.

With our general framework of participatory sensing as a provision and
appropriation system, we have a method of specifying participatory-sensing
problems in terms of these actions and the pools of information which are
generated. We have shown that within this framework which can specify
institutional rules to adhere to each of Ostrom's principles, but also that
other methods of managing the resource can be specified within the same
framework simply with a change in institutional rules and/or changes to the
institutionalised powers, permissions and obligations of certain roles.

Our implementation of this framework and rule-set therefore allows for direct
comparison between organisational paradigms, and individual rules or rule
permutations. We find with self-organising mechanisms, that inferring the
performance of combinations thereof is not always
predictable~\citep{Sanderson2013}, and therefore being able to test them in
simulation is beneficial. As our implementation is also distinct from the
experimental simulation model, it can be separated and deployed in a runtime
scenario with the same rule-set. This is a key advantage of working with an
\emph{executable} specification, that is also performant and scalable.

Our approach aims to find a method of democratising the large-scale
aggregation of user-generated information, with participatory sensing as an
example of this process. As these systems are digital, and often online, using
electronic institutions and algorithmic governance enables us to bring both
the enforcement of rules, as well as governance determining what the rules
are, into one place. It also enables autonomous agents to interact with the
governance layer along with human actors.

We see this as a tool for enabling collective data sharing in open multi-agent
systems while maintaining rights and appropriate rewards for individual
contributors. It provides a middle ground between the two extremes of
information sharing---sharing nothing and sharing everything. The former is
often the preferred strategy of self-interested agents, leading to socially
sub-optimal results. The latter is an altruistic strategy which is likely to
benefit those who don't share, more than those who do. This middle ground gives
incentives to ensure everybody contributes while still protecting the resource.

\subsection{Principled Operationalisation with Presage2}

In the simulation platform, Presage2, we have built a general purpose tool for
the agent-based modelling. We have also shown its suitability for the
principled operationalisation and controlled experimentation phases of the
\ac{SIC} methodology, both in this thesis and in other
work~\citep{Macbeth2014}.

While there exist multiple simulation platforms for multi-agent systems and
agent-based systems, we argued that an important, and oft ignored, feature of
a platform is an environment-level specification of the observability of
simulation state. This is an intermediary in between an agent and the raw
state of the environment, which can limit access to state or provide a modified view of it.
This has two advantages:

Firstly, it provides a level playing field for all agents. Particularly if
agents are being developed independently, they must agree on a level of
observability that they would expect in the simulation context. By specifying a
single \ac{API}, which both teams' agents will use it prevents any ambiguity,
or `cheating' in terms of access to simulation state.

Secondly, the underlying state representation may change, but if the \ac{API}
is constant the agent will still be able to function. The \acp{API} expressing
reading and writing of state are therefore just connectors to an outside
environment of any kind---not just a simulated one. Therefore we can take
agents out of simulation and into deployment by rewriting these \acp{API} for
interaction with the target environment.

The development of Presage2 also goes some way to addressing a limitation in
the previous Presage implementation, given by~\citet{Neville2011}. Scalability is noted as an issue for their implementation, as a single-threaded
program. Presage2 is designed to support multi-threaded simulation natively
and seamlessly. However, we have yet to explore fully distributed simulation
with the platform yet.

The Presage2 platform has shown its usefulness and generality in its use in
this thesis, as well as simulating several other models, including simulation
of resource-allocation systems, which our work is built upon (cf.
\citet{Schaumeier2013,Pitt2014}).

\subsection{Specification of Electronic Institutions}

We have shown the performance improvements available through the translation
of \ac{EC} specifications into business rules. This enables fast, real-time
deductive reasoning with executable specifications, and we showed the
application of this for electronic institutions.

Our use of the Drools-EInst engine within a Presage2 simulation allows us to
address two limitations given in the previous work of \citet{Schaumeier2013},
which used an electronic institution in a Presage2 simulation.

Firstly, in their simulations, institutionalised power was only represented implicitly,
using conditions in the agent logic. This approach precludes full listings of when
powers, permissions and obligations occur, requires that the agent know of the
conditions which lead to its empowerment, and therefore also precludes dynamic
institutionalised powers whose activating conditions change. With Drools-EInst
we explicitly represent institutionalised power, and furthermore give software
tools to allow agents to respond to the presence or absence of powers,
permissions and obligations. Thus we have a richer model of the institution,
allowing us detect events, such as when an agent is obliged to perform an
action but has yet to do so.

Secondly, their electronic institution implementation was synchronous in
simulation, in that processes which are usually asynchronous to the main
simulation task had to be resolved on the same time-scales. For example the
entirety of a voting protocol would have to be completed in a single
simulation time-step. In reality institutions have the concept of `action
situations' which can operate independently on different time-scales. In our
simulations, and with Drools- EInst, we achieved this by firstly having the
voting protocol fully specified, requiring independent actions for opening and
closing ballots with will usually occur multiple time-steps apart. Secondly,
agents are reactive to the voting process, so will be aware of when they can
vote, but not being directly compelled to do so. Thus we can simulate multiple
protocols operating asynchronously on independent time scales.


% In summary, have made two main contributions, namely:
% \begin{itemize}
% \item Data clouds in open participatory-sensing applications can be construed as information and knowledge commons and thus characterised by provision and appropriation actions; and
% \item A system for access control (i.e. provision and appropriation) in participatory-sensing applications can be designed according to Ostrom's institutional design principles for self-governing institutions and formally specified in an action language.
% \end{itemize}
% We presented in Section~\ref{sec:commons} the literature on how information and knowledge can be seen as a commons, and, being a system with the purpose of gathering information in order to generate new knowledge, that participatory sensing enables the creation of information and knowledge commons. Following a review of participatory-sensing applications we were able to characterise a generic participatory-sensing application as a provision and appropriation system in Section~\ref{sec:iad}. We then formally defined a framework for the management of an information and knowledge commons in participatory sensing.

% We have two primary motivations for why such a framework is needed in this domain. 
% Firstly, through our review of participatory sensing in Section~\ref{sec:review}, we identified a lack of governance supplied for applications. Given that participatory sensing is a knowledge commons, appropriate governance, as stressed by \possessivecite{Ostrom1990} work, is important for its successful management. The success of certain knowledge commons such as Wikipedia and Free/Open-Source Software, which have been retrospectively shown to conform to the principles which Ostrom proposed, further reinforces this point. Thus, as information and knowledge is gathered more and more on the Internet, it is important that governance is supplied in order to fairly protect the interests of all involved parties and stakeholders.

% Secondly, the digitisation of information and knowledge has caused a shift in its properties as a good~\cite{Ostrom2003}. However, legal interpretations have not yet been fully updated to this new reality, leading to exploitative and restrictive consequences~\cite{Lessig2004}. We concluded that this gives the knowledge creator complete control over what class of good their knowledge falls in to, and leads to a dilemma where knowledge is enclosed for short-term benefits and a negative long-term outcome.

% With the presentation of this framework we are taking the first steps to the supply of institutional governance for managing information and knowledge commons in participatory-sensing applications.
% What remains to do is a quantitative comparison of the different rule permutations which can be instantiated on top of our framework and an objective comparison between centralised and community governance. 
% Our framework is flexible enough to be able to represent a centralised governance as well as the self-organising approach which is required for Ostrom's principles. 
% It is the aim of our future work to answer these questions.

% In conclusion, we have shown that consideration of governance and supply thereof is important in order to achieve the potential of the participatory-sensing paradigm. 
% By drawing on the management of the commons, we can create efficient and empowering institutions around information and knowledge resources, which can exploit the power of open data. 
% Additionally, given increasing concerns over data acquisition and data privacy in the digital age, our general representation of a system for the provision and appropriation of information could have applications beyond those which we have discussed here.


\section{Limitations}

\subsection{Participatory sensing simulations}

There are several limitations in our model of participatory sensing. As with
any model, there multiple explicit and implicit assumptions and
simplifications made in its creation and usage, which will affect the
simulation results. We discuss these assumptions, how they might affect the
results, and how we might accommodate them if we wish to test them.

\paragraph{Parameters} Our simulation model has many degrees of freedom in the
parameter specification, giving too many permutations to investigate
exhaustively. We chose parameter sets to attempt to exemplify certain likely
scenarios. The choice of these parameters have a significant influence on the
results. For example, we only explore a small subset of the possible facility costs
available in our model, which may lead to interesting results, however do not
contribute towards our experimental agenda.

\paragraph{Explicit Assumptions}
We make four explicit assumptions for our simulation model (\autoref{sec:modelimpl}). These are:
\begin{enumerate}
\item Rewards gained from an action is directly measurable by the agent.
\item Rewards are a trade-able and divisible currency. This reduces complexity by preventing the need for a separate currency.
\item The only flow of rewards into the system is that generated by agents' actions on the reinforcement learning problem. For example an \emph{analyst} cannot sell information elsewhere to generate rewards.
\item Agents are working on the problem long-term, therefore we have repeated (inter)actions and thus can assume that agents are likely to be cooperative
\end{enumerate}

The first assumption is merely slightly tighter constraint on what must be true
for the model to be valid: that some kind of reward can be measured. When a
direct reward measurement is not possible, a reward can be derived implicitly
from the agent's overall progress towards a goal. If it is not possible to
measure a reward, then it is impossible to apply learning to the problem, and
there is also no sensing which can be done. Therefore the problem is neither a
reinforcement learning problem, nor a participatory-sensing one.

The second assumption is a simplifying assumption and enables frictionless
exchange of currency between agents. In real-world scenarios micropayments are
likely to come with some overhead~\citep{papaefstathiou2004}. However, recent
traction gained by digital, cryptographic currencies, such as Bitcoin\footnote{http://bitcoin.org}, have
the potential to significantly lower transaction costs, and lower the barrier
to currency creation (provided these new technologies are not subjected to
over-regulation in the future).

The third assumption prevents external rewards from flowing into the system.
Therefore all rewards are derived from strategies used on the learning
problem. Due to assumption 1, this then necessitates that gatherers are also
consumers, as if one consumes, one is also able to gather. However, there were
cases in our review of participatory-sensing applications where the gatherer
and consumer groups were disjoint or only partially intersecting. For example,
OpenSignal sells the information and knowledge in generated from collected
mobile phone coverage data to third-party companies. Having disjoint gatherer
and consumer groups removes the reciprocity with analysts, and will change the
dynamics of collective choice which we showed in our experiments. Therefore,
our model is currently unable to tell us much about these kinds of sensing
applications.

Finally, the fourth assumption has important implications for the commons. If
agents are only briefly using the commons and then will not use it ever again,
then they will have no incentive to act sustainably in their use of it. All
common-pool resource systems have the same assumption---that there is a core
of parties who are interested in the maintenance of the resource for their
long-term benefits from it. Depending on the level of churn of agents, we may
still be able to sustain an institution, but that remains to be tested.

\paragraph{Multiple Analysts}  In our simulations we use a single agent in the
role of analyst. This leads to this agent having significant power over the
institution, as shown in the results when this agent follows a greedy
strategy. To address this we may consider having multiple analysts, and this
would be a desirable scenario, giving more choice and competition to agents.

However, our simulation model is ill-equipped for this configuration. As
agents appropriate a single prediction each time-step, adding another
analyst will reduce the number of appropriations per analyst by half. This has
the implication that, in order to achieve equity, as we have defined it, the
institution must compensate analysts twice as much per prediction.
Furthermore, if one analyst's predictions are `better' (\ie\ generate higher rewards) than the others, consumers will preferentially appropriate these
predictions, and the second analyst brings no benefit to the institution.

Therefore, this poses the question of whether it is worthwhile to have the
redundancy of multiple analysts. With the simulated reinforcement-learning
problem we use for our simulations it is not, however we see several cases when
it may be:

\begin{itemize}
\item If predictions are specific to an individual consumer, and the analyst has limited computational capacity to generate them, then additional analysts will increase the capacity of the system to handle consumers' requests.
\item If analysts' knowledge is comparatively better in different areas of the solution space then combined they will provide better knowledge than either would be able to individually.
\item If analysts provision knowledge, in the form of their prediction algorithm, then the combination of multiple analysts' knowledge could lead to better knowledge as a result. 
\end{itemize}



\paragraph{Voting and counting votes} Our institutional model used preference
voting, and determined winners by a Borda count procedure. This is one of many
possible permutations of voting method. The choice of voting protocol can have
a significant effect on the result of a vote---as \citet{Pitt2011b}
demonstrate, the same preferences can lead to a different result under each
winner determination method. Furthermore, most winner determination methods
are also open to exploitation via collusion.

In our simulations we often encountered the problem of two groups with equal
voting power, representing opposite views. Therefore, the Borda count was
useful in this context to generate a fair winner in the mid-point between
these views. Other methods, such as plurality and instant runoff would result
in a deadlock in these situations.

% \paragraph{Scalability} The provision and appropriation system we have
% proposed is intended to be able to handle big data and large numbers of users,
% however the implementation with Drools-EInst is unlikely to be able to handle
% data on this scale yet. Our current implementation stores all provisioned
% information in the rule engine, which in Drools is kept in memory for fast
% rule processing. This storage limitation can be overcome by a Drools feature
% which stores facts in a database, however this again becomes a bottleneck once
% the quantity of information expands beyond what can be kept on a single hard-disk
% drive. For truly big data, information warehousing and indexing will have
% to be offloaded to distributed file-systems and databases.

\subsection{Presage2}

\paragraph{Support for Modelling} The Presage2 platform currently requires
strong programming experience in order to effectively implement models. The
platform particularly lacks graphical tools for the expression of models
compared to other popular tools such as NetLogo and RePast.

\paragraph{Modelling Libraries} As a relatively young project, Presage2 lacks
the range of modules libraries of other simulation software. This means that
users will often have to implement certain features themselves for their
models, where they may not need to if they were using other platforms.
However, this is a property of software which improves over time as community
grows around the project.

\subsection{Drools-EInst}

\paragraph{Scalability}

In \autoref{ch:droolseinst} we showed how the performance of Drools-EInst was
significantly better than that of \ac{EC} implementations, and fair when
compared to a fast, hard-coded implementation, when performing deductive tasks
on a representative electronic institution specification with large numbers of
actions. However, this test, which contained, as a maximum, thousands of
fluents and actions may not be representative of the kinds of conditions we
may expect the rule engine to handle in large-scale electronic institutions.

The Drools-EInst library depends on JBoss Drools for its performance. The
current Drools implementation keeps all state in memory and has single
threaded execution, therefore this will cause scalability issues as the number
of fluents and actions increase. Depending on the requirements of the
specification, memory usage can be managed through the removal of old facts
from the state. However, executing large numbers of facts serially may result
in performance that is too slow for real-time environments. Parallelisation of
the Rete algorithm (which Drool's engine is based on) is something which could
address this in the future.

\paragraph{Correctness} We presented no proof of the correctness of the
translation of \ac{EC} into Drools. However, we are able to use unit-testing
to provide confidence in the correctness of a translation. As the translation
process should make changes to the representation of state in the
specification, a formal proof of correctness would be difficult to achieve,
and require a more stringently prescribed process of rule and state
translation that what we have presented. Therefore we leave this as further
work.

\section{Further Work}

% \subsection{Management of Knowledge Commons}

\subsection{Further investigations with the simulation model}

Our simulation model of participatory sensing as a knowledge commons is really
just beginning to explore the issues of self-organising management of
information and knowledge. The implementation already has scope to begin
investigating various other factors around this problem which we identified at
the analysis stage in \autoref{sec:iad}.

% We firstly explore lines of work immediately available from the simulation model:

\paragraph{The evaluator role} Our experiments do not explore the role of
\emph{evaluator} in the participatory-sensing application. We envisage this
role acting their in a monitoring position, detecting users provisioning low
quality information to the pool, or as a reviewer, providing information about
the quality of other information. This tasks are complementary, so could both
be performed by the same process. Execution of these tasks will improve the
quality of information in the pool and better inform agents about which
information will be most beneficial to them. We would like to explore,
firstly, how to integrate this role into the model as a method of implementing
monitoring, and secondly, how collective evaluation of information quality
compares to individual methods, which may use trust or social
capital~\citep{Petruzzi2014} to access quality of provisions based on previous
outcomes.

\paragraph{Non-compliance} 
In our experiments we did not explore non-compliance with the institutional rules.
There exists significant scope for \emph{institutional non-compliance} within our
simulation. This encompasses performing actions without institutionalised
power and/or permission to do so, and not performing an action which one is
obliged to do.

This non-compliance has been explored in previous work, and principles
4, 5 and 6 offer mechanisms to mitigate the effect of it. However, with the
improved institutional representation offered by Drools-EInst allows for more
subtle non-compliance than has been previously tested. For example, by solving
the limitation of synchrony from previous work, we open up the possibility to
examine the effect of delayed decision making on non-compliance. For example,
if the chair of a vote expects the the result will not go in his favour, he
can decide to not declare the result---even though he is obliged to---in order to delay
the negative effect on himself. Even though this non-compliance should be
caught and punished, the delay may be sufficient to do damage to the
institution.

We could test the effect of
institutional non-compliance, by introducing a behaviour which ignores certain
obligations. This can be non-intentional non-compliance, where the agent
breaks an institutional rule because of a fault or other problem, for example
if there was a communication problem prevent the agent from being notified of
an obligation. Alternatively, malicious non-compliance is where an agent will
selectively break rules in order to maximise rewards for himself. Previously,
these kinds of compliance have been address with Ostrom's Principles 4, 5 and
6~\citep{Pitt2012b}.  

\paragraph{Knowledge pools} So far we only used pools in the institution for
information, in the form of information about how the problem works
(measurements of rewards), and information about how best to solve the problem
(predictions about the best strategy). We would like to explore the sharing of
knowledge in the commons. This could be modelled by types of knowledge,
possessed by agents and distributed amongst the population, which act as
building blocks and can be provided to create better prediction algorithms.
For example, in machine learning some algorithms work by composition of other
learning algorithms~\citep{Opitz1999}, and we can model this process such
that agents are able to iterate on a pool of knowledge to find the best
composition for their target problem.

\paragraph{Reinforcement learning problem} We used a model of a reinforcement
learning problem for our simulations in order to get determinism and
repeatability in the results. However, the use of actual learning problems and
policies would expose several other issues in the simulation. The inherent
non-determinism of reinforcement learning would have implications for equity,
both coincidently, or possibly due to malicious activity. These learning
algorithms rely on occasionally `exploring' unknown actions to find possible
better alternatives. Most times these exploration actions will give lower
rewards than the current greedy strategy choice, therefore there is a question
of how such actions should be allocated.

If this allocation is done randomly, then an `unlucky' agent could end up with
significantly lower accrued rewards purely through bad allocations. There are methods
to make allocations fairly in such scenarios (cf.~\citet{Pitt2014}), however
in this case the allocation is at the discretion of the predictor provider,
and outside the boundaries of the institution. This opens the door for
corruption in this allocation method.

Alternatively, both exploitative and exploratory predictions could be
provisioned to the pool and labelled as such, then additional incentives could
be provided to those who chose to do exploration. With proper selection of the
value of this incentive, this method would have the potential of finding an
optimal exploration rate, while socialising the cost of exploration.

Further work is required to test these issues.
If a viable solution to this problem is found we would then be able
to begin to investigate how the nature of the learning problem to be solved
affects how the institution needs to be managed. This would also further test
robustness in the presence of different levels of rewards and costs.

\paragraph{Long-term institutional change} In this work we only dealt with
scenarios which started in a certain state and then observed how this state
developed over time. We did not test how the system response to `shocks' or
other stimuli after a steady state had been reached. For example, a large
influx of agents to the system will affect the equilibrium. What is the effect
on outcomes for the original population and the newcomers? Does the system and
should the system favour those who were there at its inception, or is it
better to join later? These are questions we would like to answer in the future.

\paragraph{External Resource flows}
Our results show that, when the provision and appropriation system is taken in
isolation, it is rarely possible to achieve the ideal of free \emph{and} libre
access to the resource---the cost of the institution and incentivising users
must be covered. In other knowledge commons, these costs are covered by
exogenous utilities accrued, or by voluntary contributions to costs. However,
within the bounds of an isolated system, paid access is required to encourage the
rational agent to share. We do not yet have a model these external resource
flows, and their effect on the system equilibrium.

\paragraph{Centralised Institution} In this work we have shown that
centralisation, while good from an efficiency perspective, can lead to
inequity, and by decentralising governance we can achieve similar efficiency,
while being robust to attempts to reduce equity. While this governance is
decentralised in terms of institutionalised powers, it is still centralised in
administration. A criticism of centralisation in other fields is that it
provides a single point of failure, or a bottleneck in a system. This system
still has this flaw---all institutional state and rules are kept in the rule
engine and executed in one place.

There are several technical solutions which would enable decentralisation of
the institutional state, like those which can be used for decentralised
facilities. For example distributed version control could be used for
institutional rules, and/or peer-to-peer technology for synchronising state
between agents. However, with a distributed system, conflicts may arise, for
example if an agent applies an old version of the rules, or reasons with an
old institutional state. This mandates a conflict resolution mechanism, as
provided by Ostrom's Principle 7.

Furthermore, if we decentralise of the institution administration we must
consider the possibility of groups separating from and later rejoining the
institution. This may be unintentional, for example a break in communication
leaves a group isolated. In this case the agents should be able to continue,
and then reconcile institutional state with the rest of the system once
communication is restored (\citet{Sanderson2012} offer a protocol for such
reconciliation of institutional facts). Here, Ostrom's Principle 8 (that
activities are organised in layers of nested enterprises) will help, as having
this principle presumes that sub-groups can have reasonable levels on autonomy
within the institution.

Agents may also intentionally break away from an institution and form their
own based on the former's rule-set. This institutional `fork' is a more
extreme form of collective choice which is enabled by having open rules. This
model, applied to \ac{FOSS} repositories has been extremely successful in
recent years, allowing one to have free reign over one's own \emph{fork} of a
software repository, but also be able to push useful changes back up to the
original repository, whose maintainers can then evaluate whether accept the
changes. This model has also been adapted to \emph{federated wikis}, which
allow personal wikis to be linked in a federated, with pages forked and merged
in between them\footnote{The developer of the first wiki on the web, Ward
Cunningham, has developed such a federated wiki, called the `Smallest
Federated Wiki', according to these principles.}. The fork and merge approach
allows each individual to curate their own information and knowledge, but with
the benefit of being able to incorporate the inputs of others into their
repository.

This presents a brief summary into the possible issues and lines of inquiry opened up by the problem of entirely decentralising an institution.

\subsection{Deployment of the Knowledge Commons framework}

In addition to the lines of investigation we can take with the current
simulation model, there are other aspects of development with respect to our
knowledge commons framework.

Firstly, as a framework for socio-technical systems, we would like to explore
the interactions which the framework enables between human and computational
actors and the institutional rules. While we have designed a framework to
incentivise participation in the system on the operational level, for an
effective institution we want high participation on the collective-choice and
constitutional levels. Thus, for human actors, we envision enabling this
participation through serious games and gamification, with an emphasis on the
visualisation of the `state' of the common pool and the rule (for example
\citet{Bourazeri12} use this approach to engage SmartMeter users). 

We mentioned in \autoref{sec:iad} that the Open Mustard Seed
platform~\citep{Hardjono2014a} was a possible target for deployment of our
framework to real-world socio-technical systems. Providing solid foundations
for data control and privacy, it would allow the introduction of our framework
at the governance and information management level. Thus, moving forward, we
still this integration as a primary implementation and deployment vehicle for
the ideas expressed in this thesis.

% gamification to encourage participation in governance
% In future work, we envision enabling this interaction through
% serious games and gamification, with an emphasis on the visualisation of the `state' of the common pool and the rule
% (see \citeasnoun{Bourazeri12} for an example using SmartMeters with energy as the common pool), and through
% design contractualism, whereby design decisions and governance models are manifested both in the code and in the interface
% \cite{Pitt12}.

\subsection{Continued development of Presage2}

The Presage2 platform continues to be developed to incorporate new features.
This includes the graphical support for modelling we outlined at the end of
\autoref{ch:presage}. This support would allow for the specification of
simulation with a \ac{GUI}, by connecting parameters to modules and agents.

We have recently been exploring the use of the Groovy programming language\footnote{http://groovy-lang.org/} for
the implementation of simulation components. This is a loosely typed language
which has full integration with Java, allowing for more accessible code with
less boilerplate which, none-the-less is able to integrate with all of
Presage2's features. A simpler to write language should have the effect of
making the platform less intimidating for inexperienced programmers, and more
productive for experienced ones.

We also aim to further develop modules for the platform for common use-cases
and domains. The availability of modules speeds up the prototyping stage of
model implementation, and reduces the amount of programming required for a
working simulation.

Finally, we hope to expand the use of the tool for the deployment of agents.
We have said that our modular approach, and the abstraction layer between
agents and environment state allow for agent implementations to be taking out
of the simulation and into a real-world deployment without re-implementation.
This is something we would like to explore and test in further work.

\subsection{Continued development of Drools-EInst}

Drools-EInst, as a fairly young project has significant scope for further
improvement work. As a library of specifications for electronic institutions,
it only has the specifications described in this thesis implemented so far.
Therefore, this library should be expanded to increase the number of available
tools it provides.

Additionally, while the tool's use in simulation has been demonstrated here,
we would like to further test it in a real application. We see good candidate for
this deployment is the Open Mustard Seed project, for which the current
libraries are appropriate as a governance tool.

We identified possible future work being a proof of the translation process
from \ac{EC} to Drools specifications. This could be complimented by a
formalisation of the translation process, in the form of a transpiler (a
source-to-source compiler) to take the \ac{EC} as an input, and generate
Drools and Java source. However, as our method aims to have a readable
specification in both languages, significant work would be required to produce
machine generated code which is human-readable, and relateable to the original
\ac{EC} source.

\section{Final Remarks}
