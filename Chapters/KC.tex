
\chapter{The Knowledge Commons}\label{ch:kc}

Participatory sensing~\citep{Burke2006} is the process of leveraging user devices which are capable of various sensor measurements, to gather data in a bottom-up fashion and gain knowledge from the analysis of this data. 
It has already been applied in many varying domains, from traffic and transportation~\citep{Costa2012,Mathur2010}, environmental conditions~\citep{Hasenfratz2012,Mendez2011}, product pricing~\citep{Deng2009} to behavioural information~\citep{Miluzzo2008}.

In all of these applications, individual users or devices are gathering data which is then aggregated by a third party. Having collected this data, it is primarily this third party who reaps the benefits from the analysis of the data. The value of data gathered through these means has been estimated in the billions of dollars~\citep{Manyika2011}. While some organisations provide some return to contributors, usually in services, the equitability of this arrangement is debatable~\citep{VanDijck2009}. This has led to a call for the empowerment of users such that they can achieve a fair exchange for their data~\citep{BuckinghamShum2012}.

A key issue is the provision of tasks for participatory sensing. This is the process by which an entity determines a set of parameters to sense, and builds community and infrastructure for this purpose.
%(\citet{Burke2006} calls this user role \emph{initiator}). 
This is often a top-down process: of the applications we have mentioned the majority are centralised. This means that a single entity holds all of the collected data (often taking ownership or property rights of the data provided by others~\citep{O'Hara2010}), sets the policies regarding access to data and knowledge derived from the data, and controls how these policies are changed.

This approach is in direct contrast to how some of the most successful resources for data and knowledge have developed on the internet. 
The Open Data Institute has shown the benefits of open access to government data and that significant additional value can be generated by allowing others open and permissive access~\citep{Shadbolt2012}. 
Wikipedia is an example where individuals contributing knowledge has created an encyclopedia at a scale which would be infeasible to do in a traditional top-down manner. 
Its continued success is down to its decentralised governance which allows its users to direct the trajectory of site policy~\citep{Famiglietti2011}.

To understand the implications of how these organisational structures affect user participation and the associated benefits from the data and knowledge generated, we look at how social organisations have developed around knowledge repositories.
\citet{Ostrom2003} argue that information can be seen at a common-pool resource and thus analysed using the considerable existing literature on the commons~\citep{Hess2007}. 
Some initial investigations have been applying this approach to user-generated content~\citep{Pitt2012} which we take further here.

From her significant field work on physical commons, \citet[p.42]{Ostrom1990} outlined `the problem of supplying a new set of institutions'. 
This is the problem that in order to form an organisation someone must first provide an initial set of rules by which the organisation and its members are governed.
This is a difficult task as there are many stakeholders to satisfy under changeable conditions. Through survey of both successful and unsuccessful institutions Ostrom extracted a set of principles which were more prevalent in successful cases.

In participatory sensing we can empower users by providing them with the ability to supply institutions, and the knowledge to assess the effect of the rules and organisational structure governing sensing applications. 
We review the literature analysing knowledge commons to derive an analytic framework for institutional design of participatory sensing applications construed as a provision and appropriation system.  %This will then allow us to understand how we can provide an organisation for the management of data and knowledge for participatory sensing.
We can then derive guidelines for firstly how a self-organising participatory sensing application can function as a provision and appropriation system using a knowledge commons, and secondly how such a system can be supplied.

%Propositions/contributions:
%\begin{itemize}
%\item That participatory sensing can be seen as an information and knowledge commons and thus characterised as a provision and appropriation system.
%\item That such a system for participatory system can adhere to Ostrom's principles for enduring institutions.
%\item We can formalise this system to allow autonomous, heterogeneous agents to provision, participate in and modify an electronic institution for the management of information and knowledge resources for participatory sensing in a sustainable and equitable way.
%\end{itemize}

This paper is organised as follows: In Section~\ref{sec:commons} we introduce the theory behind information and knowledge as a commons, and how participatory sensing can be seen as such a commons. 
We then review participatory sensing applications in Section~\ref{sec:review}. This is followed by an analysis of participatory sensing as a knowledge commons according to Ostrom's institutional analysis and development framework in Section~\ref{sec:iad}. 
This analysis allows us to then create a formal representation of a system for self-organising management of a participatory sensing application as an information and knowledge commons and subsequently evaluate this system. 
Our evaluation shows that such institutions could enable autonomous, heterogeneous agents to manage information and knowledge commons used for participatory sensing applications while maintaining important criteria such as sustainability, participation standards and equity.

Based on a critical review and analysis of some representative participatory applications, we demonstrate that:
\begin{itemize}
\item Data clouds in open participatory sensing applications can be construed as information and knowledge commons and thus characterised by provision and appropriation actions; and
\item A system for access control (i.e. provision and appropriation) in participatory sensing applications can be designed according to Ostrom's institutional design principles for self-governing institutions and formally specified in an action language.
%\item That such institutions can enable autonomous, heterogeneous agents to manage information and knowledge commons used for participatory sensing applications while maintaining important criteria such as sustainability, participation standards and equity.
\end{itemize}

This provides the foundations for engineering knowledge commons for the next generation
of participatory sensing applications, in which the data generators are also the primary beneficiaries.

%Participatory sensing~\citep{Kanhere2013},~\citep{Burke2006}; Democratising big data~\citep{BuckinghamShum2012}; User agency in UGC~\citep{VanDijck2009}; lock in~\citep{Fitzpatrick2010a}; TACT~\citep{Shadbolt2013}; signing away data, data ownership is taken by a license agreement~\citep{O'Hara2010}; Value of data~\citep{Manyika2011}; Senseweb~\citep{Kansal2007}.

\section{Data and Knowledge as a Commons}\label{sec:commons}

In this section we address how participatory sensing can be seen as an information and knowledge commons. 
We give some background on the commons and introduce Ostrom's work on mangaging common-pool resources. 
We describe how the characteristics of information and knowledge as resources mean that repositories of them can also be viewed as commons, and therefore managed in the same way. % TODO: one more sentence?

%What is a commons (shared resource system) + Hardin tragedy.
The commons is a term generally used to describe shared resource systems. They are often also called common-pool resource (CPR) systems. These systems are characterised as systems where the ownership of the resource system (land, air etc.) and resource units (trees, radio frequencies, etc.) is shared, public property, or not covered by any property rights; and it is difficult to exclude access to the resource and resource units to others. 
\citet{Hardin1968} theorised that these properties would inevitably lead to overuse and depletion of the resource, a `Tragedy of the Commons', and that enclosure (via privatisation or centralisation) of the resource system was the only solution.
%Among the first areas where the commons phenomenon was identified were in fisheries and forestries. In these industries the ownership of the land (or sea) was shared or public property, the resource (fish and trees) were not owned by anyone, and it was difficult to prevent access to the resource by others. 