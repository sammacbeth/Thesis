%!TEX root =../MacbethThesis.tex
\chapter{A Rule-Engine for Electronic Institutions}\label{ch:droolseinst}

\lettrine[lines=3]{E}lectronic Institutions are a useful tool for implementing
governance in agent-based systems~\citep{Esteva2001}. Institutions are defined
as sets of rules which define what is permitted and forbidden within a given
context~\citep{Ostrom1990}. Institutions are also built with the capability to
change themselves, giving self-adaptive properties~\citep{North1990}. The
implementation of electronic institutions, therefore, has focused on the use of
rule-engines and action languages~\citep{Artikis2012a,Arcos2005,Garcia-Camino2009}.

When specifying an electronic institution we are defining rules within some
arena which determine who has \emph{institutionalised power}, \emph{permission}
and \emph{obligation} to perform certain actions~\citep{Jones1996}. In an open
system any agent could perform any action (within its physical capabilities). An
electronic institution used in the context is able to distinguish between
actions which should and should not cause a change in institutional state, and
actions which are allowed and forbidden.

A useful institution will also make agents aware of when they have the power,
permission or obligation to do an action. This aids in the planning of actions,
so that agents do not waste energy with ineffectual actions, or worse, forbidden
actions which may lead to sanctions. For example, most governments will notify
their citizens of the time window in which going to ones local polling station
and marking a cross on a piece of paper will count as an election vote. Outside
of this time window this action is, at best, a waste of time, and, at worst,
breaking and entering.

Thus, an implementation of an electronic institution should be an executable
specification which, firstly, holds some state of the institution, though which
powers, permissions and obligations can be deduced. Secondly, defines how
actions may change the state of the institution at a given time, and, finally,
allows agents to query whether they are empowered, permitted or obliged to perform
any given action.

Furthermore, in human organisations, certain useful procedures have been
formalised and specified in such a way that they can be reused in institutions
as a tool for certain tasks. For example, \ac{RONR}~\citep{Robert2011} is a
specification of a parliamentary authority for organisations. Organisations
can take this specification and update their institutional rules accordingly
to incorporate this robust protocol. The aim of electronic institutions is to
provide such tools for open, agent-based systems, and work has been done on
specifying various protocols, include the aforementioned
\ac{RONR}~\citep{Pitt2005a}, floor control
protocols~\citep{Artikis2004,Artikis2009b}, monitoring and sanctioning
procedures~\citep{Pitt2012c}, argumentation~\citep{Artikis2003} and auctions~
\citep{Rodriguez1997}.

We are concerned with how to implement electronic institutions for both
simulated applications and real-time system monitoring. This implementation
should therefore be fast-enough for real-time systems, expressive enough such
that complex rules can be implemented, readable such that the behaviour is
apparent from the specification, and, modular such that institutions can be
built up from the composition of smaller modules.

In this chapter we investigate the specification and execution of electronic
institutions. We review languages for specification of institutional rules.
Taking the Event Calculus as an example, we assess its use for the execution of
Robert's Rules of Order. We go on to show how performance can be drastically
improved using a naive rule-based implementation of the Event Calculus. Finally,
we argue that a Business rule engines have the necessary features to implement
electronic institutions, and show how Event Calculus specifications can be
translated to the JBoss Drools rule engine, using the \ac{RONR} example. 

\section{Executable Specification of Electronic Institutions}



\section{The Event Calculus}

\section{JBoss Drools}

EC: 
Combination of protocols (but slow)~\citep{Carr2012}. 
Floor control~
Provision and appropriation systems~\citep{Pitt2012}. 
Voting protocols~\citep{Pitt2005a}. 
Institutionalised consensus~\citep{Sanderson2012}. 
Resource sharing~\citep{Artikis2005}.
Monitoring and sanctioning~\citep{Pitt2012c}. 

Argumentation Protocol (in C+)~\citep{Artikis2003}.

Norms as rules with JESS~\citep{Garcia-Camino2005}.

Constraints~\citep{Aldewereld2007}.

% Comparison EC to C+ Artikis2009 specifying norm-governed comp socieities

% Languages:
% InstAL
% C+ Giunchiglia2004
% EC
% AMELI